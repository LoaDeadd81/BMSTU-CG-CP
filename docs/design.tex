\chapter{Конструкторская часть}
В этом разделе представлено описание используемых типов данных, подробный разбор алгоритмов и их схемы.


\section{Описание используемых типов данных}

Используются следующие типы данных:

\begin{itemize}[label=---]
	\item Point --- точка;
	\item Vector --- вектор;
	\item Color --- цвет;
	\item Polygon --- полигон, в нём хранятся вершины;
	\item SceneObject --- объект сцены;
	\item Sphere --- шар;
	\item Light --- точечный источник света (цвет и координаты источника света);
	\item Camera --- камера (координаты, направление и вектор, указывающий наверх);
	\item Ray --- луч, в нём хранятся его начало и направление;
	\item Scene --- сцена, хранящая все объекты.
\end{itemize}

\section{Разработка алгоритма обратной трассировки лучей}

\subsection{Алгоритм обратной трассировки лучей}
На рисунке \ref{img:trace_diag} представлена схема алгоритма обратной трассировки лучей.

\imgHeight{220mm}{trace_diag}{Схема алгоритма обратной трассировки лучей}
\clearpage


\subsection{Нахождение пересечения луча с полигоном}

Для поиска пересечения луча с треугольником используется алгоритм Моллера --- Трумбора. Пример треугольника и луча приведён на рис. \ref{img:intersect}.

\imgScale{1}{intersect}{Пример треугольника и луча}
\clearpage


\quad Введём следующие обозначения, соответствующие рисунку \ref{img:intersect}
\begin{itemize}[label=---]
	\item $P$ --- точка пересечения;
	\item $O$ --- начало луча;
	\item $\lambda$ --- расстояние от $O$ до $P$;
	\item $\vec{v}$ --- направление луча;
	\item $P_{0}$, $P_{1}$, $P_{2}$ --- вершины треугольника;
	\item $u$, $v$ --- барицентрические координаты.
	\item $b_{1} = P_{2} - P_{2}$ --- барицентрические координаты.
\end{itemize}

Барицентрические координаты представляют собой отношения площадей маленьких треугольников к большому треугольнику. Имея 3 точки на плоскости, можно выразить любую другую точку через её барицентрических координаты по формуле (\ref{eq:bar}) из определения барицентрических координат. Если каждая из этих координат будет больше или равна нулю и сумма $u + v$ будет меньше 1, то искомая точка принадлежит треугольнику. 
\begin{equation} \label{eq:bar}
	\begin{aligned}
		P(u, v) = (1 - u - v)\cdot P_{1} + u\cdot P_{2} + v\cdot P_{3}.
	\end{aligned}
\end{equation}

Выразим точку пересечения через параметрическое уравнения луча
\begin{equation} \label{eq:param}
	\begin{aligned}
		P(t) = O + \lambda\cdot \vec{v}.
	\end{aligned}
\end{equation}

Приравняв правую часть уравнений (\ref{eq:bar}) и (\ref{eq:param}), получим:
\begin{equation} \label{eq:res}
	\begin{aligned}
			P(t) = O + \lambda \cdot \vec{v} = (1 - u - v)\cdot P_{1} + u\cdot P_{2} + v\cdot P_{3}.
	\end{aligned}
\end{equation}

Уравнений \ref{eq:res} по сути является системой из 3 уравнений с тремя неизвестными $u$, $v$, $\lambda$.

Проведя алгебраические преобразования, получим

\begin{equation} \label{eq:res}
	\begin{aligned}
	\begin{bmatrix}
		\lambda\\
		u\\
		v
	\end{bmatrix}
	= \frac{1}{(\vec{D}, \vec{b_{1}})}\cdot 
	\begin{bmatrix}
		(\vec{Q}, \vec{b_{2}})\\
		(\vec{D}, \vec{T})\\
		(\vec{Q}, \vec{v})
	\end{bmatrix}.
	\end{aligned}
\end{equation}
где приняты следующие обозначения: 
\begin{itemize}[label=---]
	\item $\vec{b_{1}} = P_{2} - P_{1}$;
	\item $\vec{b_{2}} = P_{3} - P_{1}$;
	\item $\vec{T} = P - P_{1}$;
	\item $\vec{D} = (\vec{v} \times \vec{d_{2}})$;
	\item $\vec{Q} = (\vec{T} \times \vec{d_{1}})$.
\end{itemize}

\subsection{Нахождение пересечения с объемлющей оболочкой}

При трассировке лучей крайне неэффективно искать пересечения с каждым полигоном объектов каждый раз. Лучше поместить объект в объемлющую оболочку и сначала проверять пересечение с ней. Если луч не пересекает оболочку, то он не пересекает объект сцены, находящийся в оболочке, и его можно сразу отбросить. В качестве такой оболочки будет использоваться сфера в связи с простотой поиска пересечения луча и сферы.


Из параметрического уравнение луча имеем 

\begin{equation} \label{eq:dsfs}
	\begin{aligned}
		X = A + t\cdot \vec{d}.
	\end{aligned}
\end{equation}

Уравнение для точки на поверхности сферы выглядит следующим образом:

\begin{equation} \label{eq:fdff}
	\begin{aligned}
		|X - C|^2 = r^2
	\end{aligned}
\end{equation}

Подставив $X$ во второе уравнение получаем:
\begin{equation} \label{eq:a}
	\begin{aligned}
		|A + t\cdot \vec{d} - C|^2 = r^2
	\end{aligned}
\end{equation}


Обозначим $\vec{s} = A - C$, тогда

\begin{equation} \label{eq:ab}
	\begin{aligned}
		|A + t\cdot \vec{d} - C|^2 = r^2
	\end{aligned}
\end{equation}

Получилось квадратное уравнение относительно t. Дискриминант считается считается следующим образом:

\begin{equation} \label{eq:ab}
	\begin{aligned}
		D = 4\cdot ((\vec{s}, \vec{d})^2 - d^2\cdot (s^2 - r^2))  
	\end{aligned}
\end{equation}

%\begin{equation}				
%t_{1,2} = \frac{-(s, d) +- \sqrt{(s, d)^2 - d^2*(s^2 - r^2)}}{d^2}
%5\label{eq:8}
%\end{equation}

Если D < 0, то объект, находящийся в объемлющей сфере, сразу можно исключать из рассмотрения, так как луч его точно не пересекает.

\subsection{Оптимизация алгоритма трассировки лучей}

Распараллеливание алгоритмов часто используют для ускорения работы их реализаций. Алгоритм трассировки лучей отлично поддаётся распраллеливанию, поскольку каждый пиксель экрана обрабатывается независимо. Можно разбить экран на сегменты, или сектора, в виде прямоугольников, которые будут обрабатываться параллельно, независимо друг от друга. 

В программе будут использованы вспомогательные потоки, выполняющие рендер изображения сцены, и запускающий их главный поток. Последний после запуска вспомогательных потоков должен будет дождаться их завершения.

Также можно строить иерахическую структуру оболочек, что позволит отбрасывать сразу целые группы объектов, которые не пересекает данный луч. Это позволит снизить трудоёмкость алгоритма.

\section{Реализация модели освещения Уиттеда}
\subsection{Нахождение отражённого луча}

Для нахождения направление отражённого луча $\vec{R}$ необходимо знать только направления нормали $\vec{N}$ и падающего луча $\vec{L}$. Падающий вектор $\vec{L}$ можно разложить на два проекции $\vec{L_{N}}$ и $\vec{L_{P}}$, как на рисунке \ref{img:refl}. 

\pagebreak
\imgScale{1}{refl}{Разложение падающего луча}

Тогда 
\begin{equation} \label{eq:ab}
	\begin{aligned}
		\vec{L} = \vec{L_{N}} + \vec{L_{P}}.
	\end{aligned}
\end{equation}
Так как $\vec{N}$ --- единичный вектор, то длинна проекции будет равна $(\vec{L}, \vec{N})$ и
\begin{equation} \label{eq:ab}
 	\begin{aligned}
 		\vec{L_{N}} = (\vec{L}, \vec{N})\cdot \vec{N}.
 	\end{aligned}
\end{equation}
С учётом того что
\begin{equation} \label{eq:ab}
	\begin{aligned}
		{L_{P}} = \vec{L} - \vec{L_{N}},
	\end{aligned}
\end{equation}
отражённый луч можно выразить как
\begin{equation} \label{eq:ab}
	\begin{aligned}
		\vec{R} = \vec{L_{N}} - \vec{L_{P}} = 2\cdot (\vec{L}, \vec{N})\cdot \vec{N} - \vec{L}.
	\end{aligned}
\end{equation}

\subsection{Нахождение преломлённого луча}

Преломлённый луч $\vec{P}$ можно найти исходя из того факта, что падающий и преломлённый лучи лежат в одной плоскости и из закона Снелиуса, который записывается так:
\begin{equation} \label{eq:ab}
	\begin{aligned}
		sin(\alpha)\cdot n_{1} = sin(\gamma)\cdot n_{2}.
	\end{aligned}
\end{equation}
где приняты следующие обозначения:
\begin{itemize}[label=---]
	\item $\alpha$ --- угол между падающим лучом и нормалью в точке пересечения лучом объекта (точке падения);
	\item $\gamma$ --- угол между преломлённым лучом и нормалью;
	\item $n_{1}$ --- показатель преломления среды, из которой свет попадает;
	\item $n_{2}$ --- показатель преломления среды, в которую свет попадает.
\end{itemize}

Введём дополнительные обозначения $n = \frac{n_{1}}{n_{2}}$, $\vec{L}$ --- падающий луч, $\vec{N}$ --- нормаль. Можно получить уравнение для вектора преломлённого луча:

\begin{equation} \label{eq:ab}
	\begin{aligned}
		\vec{P} = n \cdot (\vec{L} + cos(\alpha) \cdot \vec{N}) - \vec{N} \cdot \sqrt{1 - sin^2(\gamma)}
	\end{aligned}
\end{equation}

Если подкоренное выражение отрицательно, то этот случай соответствует полному отражению.

\subsection{Общая реализация модели освещения Уиттеда}

Раскрыв зеркальную и диффузную составляющие в формуле (\ref{eq:whitted}) согласно модели Фонга и просуммировав их по всем источникам света, получим окончательную формулу для расчёта интенсивности:
\begin{equation} \label{eq:ab}
	\begin{aligned}
		& I = k_{a}\cdot I_{a}\cdot C + k_{d} \cdot  \sum I_{i}\cdot (\vec{n}, \vec{l_{i}}) \cdot  C + \\
		& + k_{s} \cdot  \sum I_{i}\cdot (\vec{v}, \vec{r_{i}})^s + k_{r}\cdot I_{r} + k_{t}\cdot I_{t}. \\
	\end{aligned}
\end{equation}

\section*{Вывод}
В данном разделе были рассмотрены используемые типы данных, алгоритмы обратной трассировки лучей, пересечения с объектами сцены, поиск отражённого и преломлённого луча и расчёт интенсивности согласно модели Уиттеда.
