\chapter*{Введение}
\addcontentsline{toc}{chapter}{Введение}

Компьютерная графика — область деятельности, в которой компьютеры используются с целью создания (синтеза) и изменения графических изображений. В наши дни компьютерная графика применяется во всех областях жизни человека, что связано с широким распространением и  стремительным ростом производительности вычислительных систем. 

Особенный интерес представляют алгоритмы построения реалистичных изображений. Они способны учитывать множество физических явлений, таких как отражение, преломление, прозрачность, блеск и тень, но являются крайне требовательными к ресурсам компьютера. Их скорость работы напрямую зависит от требований к качеству и реалистичности синтезируемого изображения. Необходимо найти баланс между производительностью и реалистичностью получаемого изображения. Одним из самых распространённых алгоритмов построения реалистического изображения является алгоритм трассировки лучей.

Целью работы является анализ и реализация алгоритма построения реалистичного изображения с применением трассировки лучей и глобальной моделью освещения.

Для достижения поставленной цели необходимо решить следующие задачи:
\begin{itemize}[label=---]
	\item описать визуализируемую сцену;
	\item описать существующие алгоритмы построения реалистичных изображений, текстурирования, моделей освещения и способы представления объектов сцены и выбрать подходящие для построения реалистичного изображения;
	\item разработать выбранные алгоритмы и структуры данных;
	\item реализовать все алгоритмы в виде программы с графическим интерфейсом;
	\item провести исследование быстродействия разработанного ПО в зависимости от количества параллельно работающих потоков, выполняющих рендер изображения сцены.
\end{itemize}