\chapter*{Заключение}
\addcontentsline{toc}{chapter}{Заключение}

В рамках курсового проекта было создано ПО для создания трёхмерных реалистичных изображений с использованием метода трассировки лучей, учитывающее цвета и оптические свойства объектов.

Цель, поставленная в начале лабораторной работы, была достигнута: реализован и исследован алгоритм построения реалистичного изображения с применением трассировки лучей и глобальной моделью освещения.

В ходе выполнения курсовой работы были решены все задачи:
\begin{itemize}[label=---]
	\item описана визуализиуемая сцена;
	\item описаны существующие алгоритмы построения реалистичных изображений, текстурирования, моделей освещения и способы представления объектов сцены, выбраны подходящие для построения реалистичного изображения;
	\item разработаны выбранные алгоритмы и структуры данных;
	\item все алгоритмы реализованы в виде программы с графическим интерфейсом;
	\item проведено исследование быстродействия разработанного ПО в зависимости от количества параллельно работающих потоков, выполняющих рендер изображения сцены.
\end{itemize}