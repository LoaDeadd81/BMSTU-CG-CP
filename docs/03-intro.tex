\section*{ВВЕДЕНИЕ}
\addcontentsline{toc}{section}{ВВЕДЕНИЕ}

Компьютерная графика — область деятельности, в которой вычислительные машины 	используются с целью создания, обработки и хранения графической информации.

В наши дни компьютерная графика применяется во всех областях жизни человека, что вызвано широким распространением ПК, мобильных телефонов и других устройств. Особенный интерес представляют алгоритмы построения реалистичных изображений в связи с ростом производительность процессоров, памяти и графических ускорителей. Эти алгоритмы способны учитывать множество физических ,таких как отражение, преломление, прозрачность, блеск и тень. Они являются крайне требовательными к ресурсам компьютера. Их скорость работы напрямую зависит от требований к качеству и реалистичности изображения, которое должно получиться в результате работы. Трудоемкость этих алгоритмов особенно проявляется при создании динамических сцен. Необходимо найти баланс между производительностью и реалистичностью получаемого изображения.

Цель: описать аналитический и конструкторский разделы курсовой работы по дисциплине "Компьютерная графика".

Задачи:
\begin{enumerate}
	\item[---] провести анализ cуществующих методов для построения реалистических изображений и выбрать самые подходящие;
	\item[---] описать в конструкторском разделе реализацию метода построения трёхмерной сцены и освещения.
\end{enumerate}

\pagebreak