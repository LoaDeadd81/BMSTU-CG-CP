\chapter{Технологическая часть}

В данном разделе рассмотрены средства реализации, описана структура программы, представлены листинги реализации алгоритма трассировки лучей и продемонстрирован интерфейс программы.

\section{Средства реализации}
В данной работе для реализации был выбран язык программирования $C++$ ввиду следующих причин:
\begin{itemize}[label=---]
	\item наличие большого количества библиотек для разработки приложений;
	\item возможность создания многопоточных приложений;
	\item наличие опыта программирования на данном языке.
\end{itemize}

В качестве среды разработки (IDE) была выбрана среда Clion, поскольку она содержит всю функциональность для удобной разработки ПО и в ней есть поддержка фреймворка Qt.


\section{Сведения о модулях программы}
Программа состоит из следующих	 модулей:
\begin{itemize}[label=---]
	\item Command --- модуль, реализующий паттерн <<команда>> для работы со сценой;
	\item Drawer --- модуль, хранящий <<холст>> и предоставляющий методы для рисования на нём;
	\item Error --- модуль с классами для обработки исключительных ситуаций;
	\item Manager --- модуль, хранящий сцену и предоставляет методы для работы с ней;
	\item Vec --- модуль, реализующий математические вектора;
	\item Matrix --- модуль, реализующий матрицы;
	\item Transformation --- модуль предоставляющий функции для получения матриц преобразования;
	\item GeometryPrimitives --- модуль, реализующий лучи, вершины, полигоны;
	\item Textures --- модуль, реализующий текстуры;
	\item Properies --- модуль с классами, хранящими дополнительную информацию для рендера;
	\item Loader --- модуль, предоставляющий загрузку моделей из файлов;
	\item Model --- модуль, реализующий объекты сцены;
	\item Light --- модуль, реализующий источник освещения;
	\item Camera --- модуль, реализующий камеру;
	\item Render --- модуль, реализующий алгоритм генерации изображения;
	\item Scene --- модуль, хранящий объекты сцены и предоставляющий методы для работы с ней;
	\item MainWindow --- модуль окна приложения;
	\item main.cpp --- файл, содержащий весь служебный код;
\end{itemize}

\section{Реализация алгоритмов}

В листингах \ref{lst:trace}--\ref{lst:refract} представлена реализация алгоритма трассировки лучей. Для функции трассировки лучей указаны аргументы $min\_x$ и $max\_x$, которые обозначают границу блока, который обрабатывает и отрисовывает данная функция. При многопоточном исполнении эти аргументы будут обозначать область ответственности каждого вспомогательного потока.

\newpage

\begin{center}
    \captionsetup{justification=raggedright,singlelinecheck=off}
    \begin{lstlisting}[label=lst:trace,caption=Трассировка лучей]
void RayTracingRendered::render(shared_ptr<Scene> scene, int min_x, int max_x)
{
	int height = props.s_height, width = props.s_width;
	double aspect_ratio = double(width) / height;
	double viewport_height = props.v_height;
	double viewport_width = aspect_ratio * viewport_height;
	double focal_length = props.f_len;
	Point3d origin = scene->cam()->origin;
	Vec3d right = scene->cam()->dir ^ scene->cam()->up;
	Vec3d dir_f = scene->cam()->dir * focal_length,
	v_2 = scene->cam()->up * (viewport_height / 2.0),
	h_2 = right * (viewport_width / 2.0);
	Point3d lb_corner = origin + dir_f - v_2 - h_2,
	rb_corner = origin + dir_f - v_2 + h_2,
	lt_corner = origin + dir_f + v_2 - h_2;
	Vec3d horizontal = rb_corner - lb_corner, vertical = lt_corner - lb_corner;
	Color color{0, 0, 0};
	double du = 1.0 / (width - 1), dv = 1.0 / (height - 1);
	double min_u = double(min_x) / (width - 1), cu = min_u, cv = dv;
	for (int j = 1; j < height; j++){
		for (int i = min_x; i < max_x; i++){
			color = {0, 0, 0};
			Vec3d dir = lb_corner + cu * horizontal + cv * vertical - origin;
			dir.norm();
			Ray r(origin, dir);
			
			if (emitRay(scene, r, color, props.max_depth))
				drawer->draw_pixel(i, height - j, color);
			
			cu += du;
		}
		cu = min_u;
		cv += dv;
	}
}
\end{lstlisting}
\end{center}

\newpage

\begin{center}
	\captionsetup{justification=raggedright,singlelinecheck=off}
	\begin{lstlisting}[label=lst:emit1,caption=Расчёт освещения для луча (начало)]
bool RayTracingRendered::emitRay(const shared_ptr<Scene> &scene, const Ray &r, Color &color, int depth)
{
	IntersectionData data;
	
	if (!closest_intersection(scene, r, data))
	return false;
	Color ambient, diffuse, specular, reflect, refract;
	color = {0, 0, 0};
	
	ambient = props.ambient * data.color;
	
	
	for (auto it = scene->LightsBegin(); it != scene->LightsEnd(); it++)
	{
		Vec3d l = (*it)->origin - data.p;
		
		if (closest_intersection(scene, {data.p, l}, data, EPS, 1))
		continue;
		l.norm();
		
		double n_dot_l = data.n & l;
		if (n_dot_l >= 0)
		{
			Vec3d refl = reflectedRay(-l, data.n);
			refl.norm();
			
			if ((*data.iter)->props().diffuse > 0)
			diffuse += (*it)->comp_color * n_dot_l * (*data.iter)->props().diffuse;
			if ((*data.iter)->props().specular > 0)
			specular += (*it)->comp_color * mirror_reflection(refl, -r.direction, (*data.iter)->props().shine) *
			(*data.iter)->props().specular;
		}
	}
	
	diffuse.member_mult(data.color);
	\end{lstlisting}
\end{center}


\begin{center}
	\captionsetup{justification=raggedright,singlelinecheck=off}
	\begin{lstlisting}[label=lst:emit2,caption=Расчёт освещения для луча (окончание)]
	
	if (depth <= 0)
	return true;
	
	if ((*data.iter)->props().reflective > 0)
	{
		Ray refl_ray(data.p, reflectedRay(r.direction, data.n));
		Color refl_color;
		if (emitRay(scene, refl_ray, refl_color, depth - 1))
		reflect = (*data.iter)->props().reflective * refl_color;
	}
	
	if ((*data.iter)->props().refraction > 0)
	{
		Ray refr_ray(data.p, refractedRay(r.direction, data.n, props.mi_world / data.iter->get()->props().mi));
		Color refr_color;
		if (emitRay(scene, refr_ray, refr_color, depth - 1))
		refract = (*data.iter)->props().refraction * refr_color;
	}
	
	color = ambient + diffuse + specular + reflect + refract;
	
	return true;
}
	\end{lstlisting}
\end{center}

\begin{center}
	\captionsetup{justification=raggedright,singlelinecheck=off}
	\begin{lstlisting}[label=lst:parallel,caption=Распараллеливание трассировки лучей]
		
void RenderCommand::execute(shared_ptr<Scene> scene)
{
	std::vector<std::thread> threads;
	int w_step = renderer->width() / thread_number, cur_x = 0;
	for (int i = 0; i < thread_number; i++){
		threads.emplace_back(run_thread, renderer, scene, cur_x, cur_x + w_step);
		cur_x += w_step;
	}
	for (auto &th: threads)th.join();
}
\end{lstlisting}
\end{center}

\newpage

\begin{center}
    \captionsetup{justification=raggedright,singlelinecheck=off}
    \begin{lstlisting}[label=lst:intersec,caption=Поиск ближайшего пересечения луча с объктом сцены]
bool
RayTracingRendered::closest_intersection(const shared_ptr<Scene> &scene, const Ray &r, IntersectionData &data,
double t_min,
double t_max)
{
	IntersectionData cur_data;
	data.t = t_max;
	bool flag = false;
	
	for (auto it = scene->ModelsBegin(); it != scene->ModelsEnd(); it++)
	if ((*it)->intersect(r, cur_data) && cur_data.t < data.t && cur_data.t > t_min)
	{
		flag = true;
		data = std::move(cur_data);
		data.iter = it;
	}
	
	
	return flag;
}
\end{lstlisting}
\end{center}


\begin{center}
    \captionsetup{justification=raggedright,singlelinecheck=off}
    \begin{lstlisting}[label=lst:reflect,caption=Расчёт отражённого луча]
Vec3d RayTracingRendered::reflectedRay(const Vec3d &in, const Vec3d &n)
{
	return in - 2.0 * (in & n) * n;
}
\end{lstlisting}
\end{center}

\newpage

\begin{center}
	\captionsetup{justification=raggedright,singlelinecheck=off}
	\begin{lstlisting}[label=lst:refract,caption=Расчёт преломлённого луча]
Vec3d RayTracingRendered::refractedRay(const Vec3d &v, const Vec3d &n, double mi)
{
	double cos_teta = (-v) & n, n_coef = (mi * cos_teta - std::sqrt(1 - (mi * mi * (1 - cos_teta * cos_teta))));
	return mi * v + n_coef * n;
}
	\end{lstlisting}
\end{center}

\section{Описание интерфейса}

На рисунке \ref{img:main} представлен оконный интерфейс программы. Он предоставляет возможность добавить, удалить и изменить объекты сцены. Также можно изменять настройки окружающей среды и рендера и просматривать изображение, полученное в результате работы.

\imgHeight{260pt}{main}{Интерфейс программы}

Для добавления модели необходимо выбрать в выпадающем меню пункт <<Полигональная модель>> и нажать кнопку <<Добавить>>. Появится диалоговое окно для выбора файла, из которого будет импортирована модель, как на рисунке \ref{img:import}. Модель добавиться в список объектов. Аналогично добавляются другие модели и источники света.

\imgHeight{260pt}{import}{Диалоговое окно для выбора модели}

Для получения изображения необходимо нажать кнопку <<Рендер>> и в окне просмотра появится результат как на рисунке \ref{img:render}.

\imgHeight{260pt}{render}{Результат работы программы}

В боковой панели можно выбрать один или несколько объектов и выполнить их преобразование или изменить оптические характеристики. Также есть возможность выбора цвета модели. При изменении цвета появляется диалоговое окно, как на рисунке \ref{img:color}. Выбранный цвет появится в окне предпросмотра над кнопкой <<Изменить>>.


\imgHeight{260pt}{color}{Диалоговое окно для выбора цвета}

Для наложения текстуры необходимо нажать на кнопку <<Изменить>> во вкладке <<Текстура>> и в диалоговом окне выбрать графическое изображение. Выбранная текстура появится в окне предпросмотра над кнопкой <<Изменить>>, как на рисунке \ref{img:texrender}.

\clearpage

\imgHeight{260pt}{texrender}{Рендер объекта с наложенной текстурой}

Аналогичным образом можно работать с источниками освещения и камерой. Дополнительные настройки окружающей среды и рендера находятся во вкладке <<Мир>>, как на рисунке \ref{img:settings}.
При клике на любой объект сцены боковая панель заполняется его характеристиками.

\imgHeight{260pt}{settings}{Дополнительные настройки}

\clearpage

\section*{Вывод}

Было приведено описание структуры программы, выбраны средства реализации ПО, приведены листинги кода, и продемонстрирован интерфейс
программы.
