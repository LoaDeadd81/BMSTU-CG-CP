\chapter{Экспериментальная часть}

В данном разделе приведён пример работы программы, а также проведён анализ быстродействия программы в зависимости от количества потоков. 

\section{Технические характеристики}

Технические характеристики устройства, на котором выполнялись замеры времени, представлены далее.

\begin{itemize}[label=---]
	\item Операционная система: Manjaro Linux \text{86\_64} Xfce 4.16.
	\item Оперативная память: 8 Гбайт.
	\item Процессор: 11th Gen Intel i5-1135G7 (8) @ 4.200 Гц \cite{intel}.
\end{itemize}

При тестировании ноутбук был включен в сеть электропитания. Во время тестирования ноутбук был нагружен только встроенными приложениями окружения, а также системой тестирования.

\section{Демонстрация работы программы}

На рисунке \ref{img:demo} представлен пример работы программы cо следующими объектами:

\begin{itemize}[label=---]
	\item растянутый куб с текстурой деревянного забора;
	\item прозрачный цилиндр фиолетового цвета;
	\item зеркальный шар;
	\item конус с текстурой камня;
	\item 2 точечных источника света.
\end{itemize}

\clearpage

\imgHeight{260pt}{demo}{Пример работы программы}

На рисунке \ref{img:demo2} представлен пример работы программы, демонстрирующий эффекты отражения и преломления.

\imgHeight{260pt}{demo2}{Пример работы программы с глобальной моделью освещения}

\section{Описание эксперимента}

Преимуществом алгоритма трассировки лучей является то, что он легко поддаётся распараллеливанию, так как все лучи считаются не зависимо друг от друга. 
В данной реализации сцена разделена на вертикальные полосы равного размера и каждый поток отвечает за отрисовку своего блока. Для проведения эксперимента будет использоваться сцена, как на рисунке \ref{img:demo}.

Цель эксперимента --- оценка времени работы реализации алгоритма в зависимости от количества потоков расчёта.
Результат замеров приведен в таблице \ref{tbl:time}.

\begin{table}[h]
    \begin{center}
        \begin{threeparttable}
        \captionsetup{justification=raggedright,singlelinecheck=off}
        \caption{Результаты замеров}
        \label{tbl:time}
        \begin{tabular}{|c|c|}
            \hline
			Кол-во потоков&Время, мс\\\hline
            1   & 19430 \\\hline
            2   & 11719  \\\hline
            4   & 10024  \\\hline
            8   & 6050  \\\hline
            12  & 6469  \\\hline
            16  & 6927  \\\hline
		\end{tabular}
    \end{threeparttable}
\end{center}
\end{table}

На рисунке \ref{img:chart} представлен результат замеров времени работы алгоритма трассировки лучей. По горизонтальной оси отмечено количество потоков, по вертикальной --- время в миллисекундах.

\imgHeight{100mm}{chart}{Результаты замеров}

\clearpage

\section{Вывод}

Были проведены замеры времени работы реализации алгоритма трассировки лучей. По результатам эксперимента видно, что программа затрачивает меньше всего времени на рендер сцены, когда количество выполняющих рендер потоков совпадает с количеством логических ядер процессора. Дальнейшее увеличение количества потоков не даёт сильного увеличения производительности.