\chapter{Аналитический раздел}

В этом разделе будет представлен анализ существующих способов представления объектов, алгоритмов построения реалистического изображения, текстурирования и моделей освещения.

\section{Описание трёхмерного объекта}

\subsection{Геометрические примитивы}
Примитив может быть описан некоторой функцией, принимающей параметры, например, центр и радиус для сферы или ширину, высоту и глубину для параллелепипеда. В качестве таких примитивов могут выступать любые геометрические объекты: куб, конус, пирамида, сфера или цилиндр и т.~д. Достоинствами данного метода являются простота нахождения пресечения луча и модели и малое количество информации для хранения представления объекта. К недостаткам можно отнести сложность создания реалистичных моделей из геометрических примитивов и трудность наложения текстур.

\subsection{Воксельная модель}
Двумерные модели можно описать с помощью пикселей. Аналогично можно и описывать трёхмерные модели, используя воксели, представляющие собой маленькие кубики, из которых строится модель. Достоинством данного метода являются простота реализации алгоритма трассировки лучей с воксельными моделями. К недостаткам можно отнести большой расход памяти для хранения представления объекта и низкое разрешение моделей.

\subsection{Полигональная модель}
Полигональная модель использует полигональную сетку, которая является совокупностью связанных между собой выпуклых многоугольников (полигонов), аппроксимирующих поверхность модели. Представление модели в виде многоугольников упрощает их отрисовку, или рендер. Чаще всего как полигон используются треугольник, так как он является простейшим многоугольником и все остальные многоугольники могут быть разбиты на треугольники. Данным способом можно описать объекты любой формы с хорошим разрешением и детализацией. Благодаря тому, что полигоны --- это плоские многоугольники, их легко использовать для имитации неровностей путём изменения их нормали, а также для текстурирования посредством указания текстурных координат. Достоинствами данного метода являются необходимость вычислять только координаты вершин при преобразованиях и небольшой объём данных при некоторой аппроксимации поверхности. К недостаткам можно отнести сложность алгоритмов визуализации и погрешность при аппроксимации~\cite{data}.

\subsection*{Вывод}
После анализа вышеописанных вариантов в качестве основного способа представления объектов сцены была выбрана полигональная модель. Так как в отличии от других вариантов с помощью полигональной сетки можно представить объекты любой формы с хорошим разрешением и детализацией, используя относительно небольшой объём памяти для хранения. Сложность же алгоритмов отрисовки не играет большой роли в программе, поскольку в данной работе предполагается получение реалистичного изображения, и критерий сложности алгоритма отрисовки не является главным.

\section{Алгоритм построения трёхмерного изображения}

\subsection{Алгоритм Z-буфера}

Алгоритм работает в пространстве изображения и в нём используется 2 буфера: Z-буфер (буфер глубины) и буфер кадра, размеры которых соответствуют количеству пикселей на экране. В Z-буфере находится информация о координате z для каждого пикселя, а в буфере кадра --- его интенсивность. В начале работы буферы заполняются минимальным значением координаты z и интенсивностью фона соответственно. Затем каждый многоугольник преобразуется в растровую форму и записывается в буфер кадра, при этом не производится никакого начального упорядочения.

\imgScale{0.6}{zbuf}{Демонстрация пересечения лучом объектов сцены по алгоритму Z-буфера}

В процессе работы глубина каждого нового пикселя сравнивается с глубиной, занесённой в буфер глубины. Если глубина нового пикселя меньше, то в буфер кадра заносится данные интенсивности нового пикселя, а в z-буфер --- новую координату z. Иначе данные в буферах не меняются.

Достоинствами данного алгоритма является простота его реализации и отсутствие сортировки элементов сцены. К недостаткам же можно отнести большой объём используемой памяти и трудоемкость реализации эффектов прозрачности и преломления, а также устранения ступенчатости \cite{rodgers}.

\subsection{Алгоритм обратной трассировки лучей}

Идея алгоритма обратной трассировки лучей основана на отслеживании взаимодействия отдельных лучей с объектами. Алгоритм используется для создания реалистичного освещения, отражений и теней, обеспечивающее более высокий уровень реализма по сравнению с традиционными способами рендеринга.

Из камеры выпускается луч в каждый пиксель экрана и отслеживается, куда он попадёт. Если луч пересекает объект, то рассчитывается освещение в этой точке в соответствии с моделью освещения. Интенсивность света в точке состоит из фоновой, диффузной и зеркальной составляющей. Последние два зависят от интенсивности и положения источника света. Итоговая интенсивность получается из суммы интенсивностей от всех видимых из точки пересечения источников света.

Также учитывается тень. Для этого из точки пресечения испускается теневой луч в сторону источника. Если находится пресечение с объектом между началом луча и источником света, то переходим к следующему источнику, иначе вычисляем интенсивность для этого источника в точке пересечения.

Если объект обладает отражающими свойствами, то вычисляется и испускается отражённый луч. Аналогично алгоритм работает с преломлённым лучом. 

Достоинствами данного алгоритма являются высокая реалистичность получаемого изображения и учёт таких физических явлений, как тень, преломление, отражение. Также алгоритм легко поддаётся распараллеливанию. Основным недостатком является производительность. Каждый раз необходимо просчитывать множество новых лучей, что создаёт немалую нагрузку на вычислительные устройства \cite{alg}.

\subsection*{Вывод}

Для создания реалистичного изображения лучше всего подходит алгоритм трассировки лучей, так как он даёт наиболее приближенный к реальности результат и учитывает отражения, прозрачность и тени.

\section{Модель освещения трёхмерных объектов}

\subsection{Модель Ламберта}

Простейшая модель освещения: она учитывает только диффузное освещение. Считается, что свет падающий в точку, одинакового рассеивается по всем направлением полупространства. Таким образом, освещенность в точке определяется только плотностью света в точке поверхности, а она линейно зависит от косинуса угла падения и считается по формуле
\begin{equation} \label{eq:lambert}
	\begin{aligned}
		I = k_{d} \cdot (\vec{n}, \vec{l}),
	\end{aligned}
\end{equation}
где приняты следующие обозначения:
\begin{itemize}[label=---]
	\item $k_{d}$ --- коэффициент диффузного отражения;
	\item $\vec{n}$ ---  нормаль в точке пресечения;
	\item $\vec{l}$ --- единичный вектор, направленный к источнику света.
\end{itemize}

\subsection{Модель Фонга}

Модель расчёта освещения трёхмерных объектов, в том числе полигональных моделей и примитивов, а также метод интерполяции освещения по всему объекту. Это локальная модель освещения, то есть она учитывает только свойства заданной точки и источников освещения, игнорируя эффекты рассеивания, линзирования, отражения от соседних тел. Эта модель состоит из диффузной составляющей и зеркальной и рассчитывается по формуле~(\ref{eq:fong}). Благодаря зеркальной составляющей на объектах появляются блики. Интенсивность в точке зависит от того, насколько близок отражённый вектор к вектору, направленному из точки падения в сторону наблюдателя. В модели учитывается интенсивность фонового, диффузного и зеркального освещения. Для расчёта диффузной составляющей используется модель Ламберта \cite{light}. Освещение рассчитывается по формуле

\begin{equation} \label{eq:fong}
	\begin{aligned}
		I = k_{a} \cdot I_{a} + k_{d} \cdot (\vec{n}, \vec{l}) + k_{s} \cdot (\vec{v}, \vec{r})^s,
	\end{aligned}
\end{equation}
где приняты следующие обозначения:
\begin{itemize}[label=---]
	\item $k_{d}$ --- коэффициент диффузного отражения;
	\item $k_{s}$ --- коэффициент зеркального отражения;
	\item $k_{a}$ --- коэффициент рассеянного отражения;
	\item $I_{a}$ --- интенсивность фонового освещения;
	\item $\vec{n}$ ---  нормаль в точке пересечения;
	\item $\vec{l}$ --- единичные вектор, направленный к источнику света;
	\item $\vec{v}$ --- единичные вектор, направленный к наблюдателю;
	\item $\vec{r}$ --- единичные вектор, отражение  $\vec{l}$;
	\item $s$ --- степень, аппроксимирующая пространственное распределение зеркально отраженного света.
\end{itemize}


\subsection{Модель Уиттеда}

Модель освещенности Уиттеда является одной из самых распространенных и наиболее часто используемой моделью в методе трассировки лучей. Использует для расчёта интенсивности глобальную модель освещения, учитывающую свет провзаимодействовавший с другими объектами. Помимо учёта фоновой, диффузной и зеркальной компонент в этой модели ещё учитывается интенсивность отражённого и преломлённого света от других тел~\cite{porev}. Освещение рассчитывается по формуле
\begin{equation} \label{eq:whitted}
	\begin{aligned}
		I = k_{a} \cdot I_{a} \cdot C + k_{d} \cdot I_{d} \cdot C + k_{s} \cdot I_{s} + k_{r} \cdot I_{r} + k_{t} \cdot I_{t},
	\end{aligned}
\end{equation}
где приняты следующие обозначения:
\begin{itemize}[label = ---]
	\item $k_{a}$ --- коэффициент  фоновой подсветки;
	\item $k_{d}$ --- коэффициент диффузного рассеивания;
	\item $k_{s}$ --- коэффициент зеркальности;
	\item $k_{r}$ --- коэффициент отражения;
	\item $k_{t}$ --- коэффициент прозрачности;
	\item $I_{a}$ --- интенсивность фоновой подсветки;
	\item $I_{d}$ --- интенсивность, учитываемая для диффузного рассеивания;
	\item $I_{s}$ --- интенсивность, учитываемая для зеркальности;
	\item $I_{r}$ --- интенсивность излучения, приходящего по отраженному лучу;
	\item $I_{t}$ --- интенсивность излучения, приходящего по преломленному лучу;
	\item $C$ --- цвет исходного объекта.
\end{itemize}

Для расчёта локальной интенсивности т.~е. фоновой, диффузной и зеркальной составляющей используется модель Фонга. 

\subsection*{Вывод}

Для получения наиболее реалистических изображений была выбрана модель Уиттеда как самая подходящая и реалистичная, так как она учитывает такие эффекты, как отражение, прозрачность, преломление, тень.